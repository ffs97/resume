%-------------------------------------------------------------------------------
%	SECTION TITLE
%-------------------------------------------------------------------------------
\cvsection{Projects}

%-------------------------------------------------------------------------------
%	CONTENT
%-------------------------------------------------------------------------------

\cventry
    {Pairwise Approach to Causal Discovery}
    {Dr.\ Alp Kucukelbir (Fero Labs)}
	{MS Thesis, CU}
	{Aug 2020 --- Dec 2020}
	{%
		\begin{cvitems}
            \item Explored the use of Xi correlations for causal discovery which proved to be a fast and effective way to discover structure in causal graphs
            \item Experimented with a measure of variance for conditional distributions (CDS) to orient causal edges on simulated and real-world data
            \item Explored strengths and weaknesses of pairwise methods for causal discovery along with presenting intuition behind the workings of CDS
		\end{cvitems}
	}

%---------------------------------------------------------
\cventry
    {Optimal Interventions for Active Causal Discovery}
    {Prof.\ Elias Bareinboim (CU)}
	{COMS 4995: Causal Inference, CU}
	{Feb 2020 --- Apr 2020}
	{%
		\begin{cvitems}
            \item Studied the common assumptions of Causal Discovery (CD), and surveyed recent advances in CD and Active CD
            \item Proposed an extension to an existing approach for finding the optimal intervention in Active CD using Bayesian Optimization to map the space of causal graphs
		\end{cvitems}
	}

%---------------------------------------------------------
\cventry
    {Bayesian ML for Predicting Underground Water Levels}
    {Dr.\ Alp Kucukelbir (Fero Labs)}
    {COMS 6998: Probabilisitic Programming, CU}
    {Sep 2019 --- Nov 2019}
    {%
        \begin{cvitems}
            \item Modeled underground water levels in a sub-region in Rajasthan (India) based on readings from sparse observatory wells
            \item Used a HMM-modulated kernel regression to model temporal and spatial patterns dependent on satellite observations of farmland
        \end{cvitems}
    }

%---------------------------------------------------------
\cventry
	{Implementing Adaptive Neural Networks}
	{Dr.\ Satyen Kale (Google Research)}
	{COMS 4995: Optimization Methods, CU}
	{Sep 2019 --- Nov 2019}
	{%
		\begin{cvitems}
			\item Implemented \href{https://arxiv.org/abs/1607.01097}{AdaNet} using PyTorch and analyzed its performance with different settings of hyperparameters
			\item Improved hyperparameter sensitivity by adaptively changing the subnetwork width without compromising on performance
		\end{cvitems}
	}

%---------------------------------------------------------
\cventry
	{Discrete Variational Autoencoders and Stochastic Block Models}
    {Prof.\ Piyush Rai (IITK)}
	{Undergraduate Project, IITK}
	{Aug 2018 --- Jan 2019}
	{%
		\begin{cvitems}
			\item Surveyed continuous relaxations to discrete latent variables and implemented GumBolt relaxation for RBM prior using Tensorflow
			\item Augmented GVAEs with binary latent embeddings to offer interpretable latent representations, imitating mixed membership models
			\item Employed the resultant model for link prediction on graph datasets (Citeseer and Cora) and achieved superior results to baseline models
		\end{cvitems}
	}

%---------------------------------------------------------
\cventry
	{Mixture of Experts using Discrete VAE}
	{Fall 2018}
	{Prof.\ Arnab Bhattacharya}
	{CS685: Data Mining}
	{%
		\begin{cvitems}
            \item Proposed a novel model using the VAE framework for clustering in latent space, extending the ideas of the VaDE model
            \item Modeled the cluster assignment using a deep neural network, and added regularization using Virtual Adversarial training
            \item The proposed model worked comparable to VaDE on clustering tasks without the need for careful layer wise pretraining
            \item Extended the proposed model as a gating function for Mixture of Experts tasks and achieved better performance than naive baseline models
		\end{cvitems}
	}

%---------------------------------------------------------
\cventry
	{CatTalks: A Centralized Video + Text Chat Tool}
	{Fall 2018}
	{Prof.\ Dheeraj Sanghi}
	{CS425: Computer Networks}
	{%
		\begin{cvitems}
            \item Developed a web-app for text and video chat using Flask and socket programming on python
            \item Login credentials, requests and messages were stored using MongoDB
		\end{cvitems}
	}

%---------------------------------------------------------
\cventry
	{Incremental Neural Networks Training}
	{Spring 2018}
	{Prof.\ Purushottam Kar}
	{CS777: Statistical and Algorithmic Learning Theory}
	{%
		\begin{cvitems}
            \item Two layer NNs can be represented as an ensemble of multiple single node hidden layer networks, which can be individually trained using generic boosting methods (gradient boosting), which also afford definite theoretical convergence guarantees
            \item Applied gradient boosting to train two layer networks incrementally and studied the convergence analysis under various constraints
            \item Implemented incremental NN training in python using sklearn, and applied for Softmax Regression on the MNIST Dataset
            \item Applied incremental training as pre-training, along with backpropagation for fine-tuning and observed remarkably better convergence
		\end{cvitems}
	}

%---------------------------------------------------------
\cventry
	{Survey on Methods For Convex Optimization}
	{Spring 2018}
	{Prof.\ Purushottam Kar}
	{CS777: Statistical and Algorithmic Learning Theory}
	{%
		\begin{cvitems}
            \item Surveyed prominent Gradient Descent based techniques (SGD, AdaGrad, etc.) for optimization and perused the convergence bounds of each
            \item Reviewed and paraphrased a paper which disproves guaranteed convergence of Adam for even convex objectives using a counterexample
            \item Identified inconsistencies within the convergence proof for Adam as an attempt to explain its incorrectness
		\end{cvitems}
	}

%---------------------------------------------------------
\cventry
	{Clustering and MoE for Arbitrary Shaped Clusters}
	{Spring 2018}
	{Prof.\ Piyush Rai}
	{CS698X: Bayesian Modelling and Inference}
	{%
		\begin{cvitems}
            \item Studied VAEs and surveyed clustering models (iWMM, SVAE, VaDE, etc.) for data existing in non-Gaussian shaped clusters
            \item Implemented Variational Deep Embeddings (VaDE) in Tensorflow to experiment on MNIST and spiral dataset to learn arbitrary shaped clusters
            \item Proposed gating functions based on VaDE and Stick Breaking-VAE for mixture of experts models
		\end{cvitems}
	}

%---------------------------------------------------------
\cventry
	{Java to X86 Assembly Compiler}
	{Spring 2018}
	{Prof.\ Subhajit Roy}
	{CS335: Compiler Design}
	{%
		\begin{cvitems}
            \item Developed an end-to-end compiler in node.js for a subset of Java language to compile into x86 Assembly using jison (for parsing)
            \item Implemented advanced features such as classes and type casting, along with support for floats. Adjudged one of the best projects
		\end{cvitems}
	}

%---------------------------------------------------------
\cventry
	{Machine Comprehension using Match-LSTM}
	{Spring 2018}
	{Prof.\ Harish Karnick}
	{CS671: Natural Language Processing}
	{%
		\begin{cvitems}
            \item Surveyed various models for Machine Comprehension (FastQA, R-Net, Match-LSTM, etc.) and implemented Match-LSTM using Tensorflow
            \item Experimented with SQuAD and combated inefficiency of Match-LSTM to apply separate attention mechanisms for different question types
            \item Additionally, introduced simple changes to loss function to improve the EM score on SQuAD by a total of over 5\%
		\end{cvitems}
	}

%---------------------------------------------------------
\cventry
	{Scaling Recommendation Systems using K-Means Clustering}
	{Fall 2017}
	{Prof.\ Purushottam Kar}
	{CS771: Introduction to Machine Learning}
	{%
		\begin{cvitems}
            \item Used K-Means clustering to divide users into cliques, and applied Collaborative Filtering independently within each clique
            \item Clustered songs based on MFCC features using K-Means and quantified user features based on song clusters from the user’s learning history
            \item Applied the model on MSD. Also proposed simple exploration strategy based on song clusters to allow variations in suggestions provided
		\end{cvitems}
	}

%---------------------------------------------------------
\cventry
	{NachOS Operating System}
	{Fall 2017}
	{Prof.\ Mainak Chaudhuri}
	{CS330: Operating Systems}
	{%
		\begin{cvitems}
            \item Implemented basic operating system functions (Fork, Join, etc.) on a truncated NachOS code (provided) in C++ programming language
            \item Implemented and evaluated performance of algorithms for various scheduling processes and various page replacement strategies
            \item Implemented Shared Memory Allocation, Demand Paging and various Page Replacement Algorithms
		\end{cvitems}
	}

%---------------------------------------------------------
\cventry
	{Smart Image Advertising}
	{Fall 2016}
    {}
	{Google DevFest}
	{%
		\begin{cvitems}
            \item Developed a web-app for smart advertising using image analysis with basic controls
            \item Wrote a back-end program to detect objects in an image on upload using an API service from Clarifai
            \item Stored objects as tags in database and used these to search through products on different e-commerce websites using their affiliate APIs
		\end{cvitems}
	}

%-------------------------------------------------------------------------------
\cvcomment{* Code and reports for all projects available at \underline{\href{https://www.github.com/fat-fighter}{github://fat-fighter}}}
